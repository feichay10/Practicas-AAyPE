\documentclass[11pt]{report}

% Paquetes y configuraciones adicionales
\usepackage{graphicx}
\usepackage[export]{adjustbox}
\usepackage{caption}
\usepackage{float}
\usepackage{titlesec}
\usepackage{geometry}
\usepackage[hidelinks]{hyperref}
\usepackage{titling}
\usepackage{titlesec}
\usepackage{parskip}
\usepackage{wasysym}
\usepackage{tikzsymbols}
\usepackage{fancyvrb}
\usepackage{xurl}
\usepackage{hyperref}
\usepackage{listings}
\usepackage{xcolor}
\usepackage[spanish]{babel}

\newcommand{\subtitle}[1]{
  \posttitle{
    \par\end{center}
    \begin{center}\large#1\end{center}
    \vskip0.5em}
}

% Configura los márgenes
\geometry{
  left=2cm,   % Ajusta este valor al margen izquierdo deseado
  right=2cm,  % Ajusta este valor al margen derecho deseado
  top=3cm,
  bottom=3cm,
}

% Configuración de los títulos de las secciones
\titlespacing{\section}{0pt}{\parskip}{\parskip}
\titlespacing{\subsection}{0pt}{\parskip}{\parskip}
\titlespacing{\subsubsection}{0pt}{\parskip}{\parskip}

% Redefinir el formato de los capítulos y añadir un punto después del número
\makeatletter
\renewcommand{\@makechapterhead}[1]{%
  \vspace*{0\p@} % Ajusta este valor para el espaciado deseado antes del título del capítulo
  {\parindent \z@ \raggedright \normalfont
    \ifnum \c@secnumdepth >\m@ne
        \huge\bfseries \thechapter.\ % Añade un punto después del número
    \fi
    \interlinepenalty\@M
    #1\par\nobreak
    \vspace{10pt} % Ajusta este valor para el espacio deseado después del título del capítulo
  }}
\makeatother

% Configura para que cada \chapter no comience en una pagina nueva
\makeatletter
\renewcommand\chapter{\@startsection{chapter}{0}{\z@}%
    {-3.5ex \@plus -1ex \@minus -.2ex}%
    {2.3ex \@plus.2ex}%
    {\normalfont\Large\bfseries}}
\makeatother

% Configurar los colores para el código
\definecolor{codegreen}{rgb}{0,0.6,0}
\definecolor{codegray}{rgb}{0.5,0.5,0.5}
\definecolor{codepurple}{rgb}{0.58,0,0.82}
\definecolor{backcolour}{rgb}{0.95,0.95,0.92}

% Configurar el estilo para el código
\lstdefinestyle{mystyle}{
  backgroundcolor=\color{backcolour},   
  commentstyle=\color{codegreen},
  keywordstyle=\color{magenta},
  numberstyle=\tiny\color{codegray},
  stringstyle=\color{codepurple},
  basicstyle=\ttfamily\footnotesize,
  breakatwhitespace=false,         
  breaklines=true,                 
  captionpos=b,                    
  keepspaces=true,                 
  numbers=left,                    
  numbersep=5pt,                  
  showspaces=false,                
  showstringspaces=false,
  showtabs=false,                  
  tabsize=2
}

% Configurar el estilo para el código en C++ y CUDA con colores
\lstset{
  style=mystyle,
  language=C++,
  morekeywords={__global__, __device__, __host__},
  deletekeywords={__global__, __device__, __host__}
}

%==============================================================================
% Cosas para la documentación LateX
% % Sangría
% \setlength{\parindent}{1em}Texto

% % Quitar sangría
% \noindent

% % Punto
% \CIRCLE \ \ \textbf{Texto} \emph{algo}
% \begin{itemize}
%   \item \textbf{Negrita:} Texto
%   \item \textbf{Negrita:} Texto
% \end{itemize}

% % Introducir código
% \begin{center}
%   \begin{BVerbatim}
%     ... Código
%   \end{BVerbatim}
% \end{center}

% Poner una imagen
% \begin{figure}[H]
%   \centering
%   \includegraphics[scale=0.55]{img/}
%   \caption{Exportación de la base de datos en formato sql}
%   \label{fig:exportación de la base de datos en formato sql}
% \end{figure}

% Poner dos imágenes
% \begin{figure}[H]
%   \begin{subfigure}{0.5\textwidth}
%     \centering
%     \includegraphics[scale=0.45]{img/}
%     \caption{Texto imagen 1}
%   \end{subfigure}%
%   \begin{subfigure}{0.5\textwidth}
%     \centering
%     \includegraphics[scale=0.45]{img/}
%     \caption{Texto imagen 2}
%   \end{subfigure}
%   \caption{Texto general}
% \end{figure}

% % Poner una tabla
% \begin{table}[H]
%   \centering
%   \begin{tabular}{|c|c|c|c|}
%     \hline
%     \textbf{Campo 1} & \textbf{Campo 2} & \textbf{Campo 3} & \textbf{Campo 4} \\ \hline
%     Texto & Texto & Texto & Texto \\ \hline
%     Texto & Texto & Texto & Texto \\ \hline
%     Texto & Texto & Texto & Texto \\ \hline
%     Texto & Texto & Texto & Texto \\ \hline
%   \end{tabular}
%   \caption{Nombre de la tabla}
%   \label{tab:nombre de la tabla}
% \end{table}

% % Poner codigo de un lenguaje a partir de un archivo
% \lstset{style=mystyle}
% The next code will be directly imported from a file
% \lstinputlisting[language=Python]{code.py}

%==============================================================================

\begin{document}

% Portada del informe
\title{Proyecto final Histograma realizado con CUDA}
\subtitle{Arquitecturas Avanzadas y de Propósito Específico}
\author{Cheuk Kelly Ng Pante (alu0101364544@ull.edu.es)}
\date{\today}

\maketitle

\pagestyle{empty} % Desactiva la numeración de página para el índice

% Índice
\tableofcontents

% Nueva página
\cleardoublepage

\pagestyle{plain} % Vuelve a activar la numeración de página
\setcounter{page}{1} % Reinicia el contador de página a 1

% Secciones del informe
% Capitulo 1
\chapter{Introducción}
Consiste realizar un histograma de un vector V de un número elevado N de 
elementos enteros aleatorios. El histograma consiste en un vector H que tiene M 
elementos que representan “cajas”. En cada caja se cuenta el número de veces que
ha aparecido un elemento del vector V con el valor adecuado para asignarlo a esa 
caja (normalmente cada caja representa un rango o intervalo de valores). En nuestro 
caso, para simplificar la asignación del elemento de V a su caja correspondiente 
del histograma, vamos a realizar la operación ValorElementoV Módulo M, que nos da 
directamente el índice de la caja del histograma a la que pertenecerá ese elemento
y cuyo contenido deberemos incrementar. Se sugiere como N un valor del orden de 
millones de elementos y como M, 8 cajas. 

% Capitulo 2
\chapter{Desarrollo del proyecto}
\section{Implementación base}
Como implementación base se pide crear tantos hilos como elementos V para cada uno
se encargue de ir al elemento que le corresponda V e incremente la caja correcta en
el vector histograma H (posiblemente de forma atómica).

El código base es el siguiente:
\begin{lstlisting}
__global__ void histogram() {
  int i = blockIdx.x * blockDim.x + threadIdx.x;
  if (i < N) {
    atomicAdd(&vector_H[vector_V[i] % M], 1);
  }
}
\end{lstlisting}

La implementación base lo que hace es crear un único histogram compartido por todos
los hilos. Cada hilo se encarga de incrementar la caja correspondiente al elemento
V que le corresponda. Para ello, se utiliza la función atomicAdd para incrementar
de forma atómica el valor de la caja correspondiente al elemento V. La implementación
se encuentra en el archivo \textit{histogram\_1.cu}.

\section{Segunda implementación}
Como segunda implementación, se va a dividir la operación en dos fases. En la primera,
en lugar de trabajar sobre un único histograma global, repartiremos el cálculo realizando 
un cierto número de histogramas que llamaremos "locales", cada uno calculado sobre una 
parte del vector de datos. La idea es reducir el número de hilos que escriben sobre la
misma posición del histograma, ya que dicha operación debe ser atómica y se serializan 
dichos accesos. La segunda fase realizará la suma de los histogramas locales en un único
histograma global final. Se debe intentar llevar a cabo esta suma de la forma más paralela
o eficiente, posiblemente utilizando el método de reducción. La implementación se encuentra
en el archivo \textit{histogram\_2.cu}.

El código es el siguiente:
\begin{lstlisting}
  __global__ void histogram() {
  // Declarar memoria compartida para el histograma local
  extern __shared__ int local_histogram[];

  // Inicializar el histograma local en memoria compartida
  int tid = threadIdx.x;
  for (int i = tid; i < M; i += blockDim.x) {
    local_histogram[i] = 0;
  }
  __syncthreads();

  // Calcular el histograma local
  int i = blockIdx.x * blockDim.x + tid;
  if (i < N) {
    atomicAdd(&local_histogram[vector_V[i] % M], 1);
  }
  __syncthreads();

  // Realizar la reduccion de los histogramas locales en un unico histograma global final
  for (int j = tid; j < M; j += blockDim.x) {
    atomicAdd(&vector_H[j], local_histogram[j]);
  }
}
\end{lstlisting}

La segunda implementación se divide en dos fases. 


% Capitulo 3
\chapter{Pruebas realizadas}
Durante las pruebas realizadas, se he ejecutado cada implementación 10000 veces para 
obtener un tiempo promedio significativo de tiempo. También, se ha obtenido el tiempo
máximo y mínimo de la ejecución. Además, se 


\chapter{Bibliografía} % En formato APA
\begin{enumerate}
\item Ng Pante, C. (2001). Titulo. Nombre pagina web. Recuperado de \url{http://url.com}

\end{enumerate}

\end{document}