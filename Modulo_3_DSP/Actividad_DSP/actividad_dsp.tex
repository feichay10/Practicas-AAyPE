\documentclass[11pt]{report}

% Paquetes y configuraciones adicionales
\usepackage{graphicx}
\usepackage[export]{adjustbox}
\usepackage{caption}
\usepackage{float}
\usepackage{titlesec}
\usepackage{geometry}
\usepackage[hidelinks]{hyperref}
\usepackage{titling}
\usepackage{titlesec}
\usepackage{parskip}
\usepackage{wasysym}
\usepackage{tikzsymbols}
\usepackage{fancyvrb}
\usepackage{xurl}
\usepackage{hyperref}
\usepackage{subcaption}
\usepackage{multicol}
\usepackage{listings}
\usepackage{xcolor}
\usepackage{eurosym} % Paquete necesario
\usepackage[spanish]{babel}

\newcommand{\subtitle}[1]{
  \posttitle{
    \par\end{center}
    \begin{center}\large#1\end{center}
    \vskip0.5em}
}

% Configura los márgenes
\geometry{
  left=2cm,   % Ajusta este valor al margen izquierdo deseado
  right=2cm,  % Ajusta este valor al margen derecho deseado
  top=3cm,
  bottom=3cm,
}

% Configuración de los títulos de las secciones
\titlespacing{\section}{0pt}{\parskip}{\parskip}
\titlespacing{\subsection}{0pt}{\parskip}{\parskip}
\titlespacing{\subsubsection}{0pt}{\parskip}{\parskip}

% Redefinir el formato de los capítulos y añadir un punto después del número
\makeatletter
\renewcommand{\@makechapterhead}[1]{%
  \vspace*{0\p@} % Ajusta este valor para el espaciado deseado antes del título del capítulo
  {\parindent \z@ \raggedright \normalfont
    \ifnum \c@secnumdepth >\m@ne
        \huge\bfseries \thechapter.\ % Añade un punto después del número
    \fi
    \interlinepenalty\@M
    #1\par\nobreak
    \vspace{10pt} % Ajusta este valor para el espacio deseado después del título del capítulo
  }}
\makeatother

% Configura para que cada \chapter no comience en una pagina nueva
\makeatletter
\renewcommand\chapter{\@startsection{chapter}{0}{\z@}%
    {-3.5ex \@plus -1ex \@minus -.2ex}%
    {2.3ex \@plus.2ex}%
    {\normalfont\Large\bfseries}}
\makeatother

% Configurar los colores para el código
\definecolor{codegreen}{rgb}{0,0.6,0}
\definecolor{codegray}{rgb}{0.5,0.5,0.5}
\definecolor{codepurple}{rgb}{0.58,0,0.82}
\definecolor{backcolour}{rgb}{0.95,0.95,0.92}

% Configurar el estilo para el código
\lstdefinestyle{mystyle}{
  backgroundcolor=\color{backcolour},   
  commentstyle=\color{codegreen},
  keywordstyle=\color{magenta},
  numberstyle=\tiny\color{codegray},
  stringstyle=\color{codepurple},
  basicstyle=\ttfamily\footnotesize,
  breakatwhitespace=false,         
  breaklines=true,                 
  captionpos=b,                    
  keepspaces=true,                 
  numbers=left,                    
  numbersep=5pt,                  
  showspaces=false,                
  showstringspaces=false,
  showtabs=false,                  
  tabsize=2
}

%==============================================================================
% Cosas para la documentación LateX
% % Sangría
% \setlength{\parindent}{1em}Texto

% % Quitar sangría
% \noindent

% % Punto
% \CIRCLE \ \ \textbf{Texto} \emph{algo}
% \begin{itemize}
%   \item \textbf{Negrita:} Texto
%   \item \textbf{Negrita:} Texto
% \end{itemize}

% % Introducir código
% \begin{center}
%   \begin{BVerbatim}
%     ... Código
%   \end{BVerbatim}
% \end{center}

% Poner una imagen
% \begin{figure}[H]
%   \centering
%   \includegraphics[scale=0.55]{img/}
%   \caption{Exportación de la base de datos en formato sql}
%   \label{fig:exportación de la base de datos en formato sql}
% \end{figure}

% Poner dos imágenes
% \begin{figure}[H]
%   \begin{subfigure}{0.5\textwidth}
%     \centering
%     \includegraphics[scale=0.45]{img/}
%     \caption{Texto imagen 1}
%   \end{subfigure}%
%   \begin{subfigure}{0.5\textwidth}
%     \centering
%     \includegraphics[scale=0.45]{img/}
%     \caption{Texto imagen 2}
%   \end{subfigure}
%   \caption{Texto general}
% \end{figure}

% % Poner una tabla
% \begin{table}[H]
%   \centering
%   \begin{tabular}{|c|c|c|c|}
%     \hline
%     \textbf{Campo 1} & \textbf{Campo 2} & \textbf{Campo 3} & \textbf{Campo 4} \\ \hline
%     Texto & Texto & Texto & Texto \\ \hline
%     Texto & Texto & Texto & Texto \\ \hline
%     Texto & Texto & Texto & Texto \\ \hline
%     Texto & Texto & Texto & Texto \\ \hline
%   \end{tabular}
%   \caption{Nombre de la tabla}
%   \label{tab:nombre de la tabla}
% \end{table}

% % Poner codigo de un lenguaje a partir de un archivo
% \lstset{style=mystyle}
% The next code will be directly imported from a file
% \lstinputlisting[language=Python]{code.py}

% “Texto entre comillas dobles”

% • 

%==============================================================================

\begin{document}

% Portada del informe
\title{Actividad sobre DSP}
\subtitle{Arquitecturas Avanzadas y de Propósito Específico}
\author{Cheuk Kelly Ng Pante (alu0101364544@ull.edu.es)}
\date{\today}

\maketitle

\pagestyle{empty} % Desactiva la numeración de página para el índice

% Índice
\tableofcontents

% Nueva página
\cleardoublepage

\pagestyle{plain} % Vuelve a activar la numeración de página
\setcounter{page}{1} % Reinicia el contador de página a 1

% Secciones del informe
% Capitulo 1
\chapter{Introducción a los DSP}
Un procesador de señales digitales o DSP es un sistema basado en un procesador o
microprocesador que posee un conjunto de instrucciones, un hardware y un software
optimizado para aplicaciones que requieren operaciones numéricas a muy alta velocidad.

Algunos fabricantes de DSP son:
\begin{multicols}{2}
      \begin{itemize}
            \item Analog Devices
            \item Texas Instruments
            \item NXP
            \item STMicroelectronics
      \end{itemize}

      \columnbreak

      \begin{itemize}
            \item Infineon
            \item Qualcomm
            \item Broadcom
      \end{itemize}
\end{multicols}

% Capitulo 2
\chapter{Automatización industrial}
La automatización industrial es la aplicación de tecnologías para operar procesos
y maquinaria sin la intervención humana. Se utiliza en una gran variedad de industrias,
como la automotriz, la robótica, la aeroespacial, la naval, la de empaquetado, la de
alimentos y bebidas, la de productos farmacéuticos y la de fabricación de metales.

Algunas aplicaciones de la automatización industrial son: control de procesos, robots industriales y colaborativos,
sistemas de transporte, monitoreo y control, sistemas de prueba, sistemas de seguridad, etc.

\section{DSP en la automatización industrial}
Los DSP (Procesadores Digitales de Señal) son dispositivos que realizan operaciones matemáticas
sobre señales digitales, como audio, vídeo o datos. Se utilizan en muchos campos de la automatización
industrial, como el control de procesos, la comunicación, el filtrado o la conversión de datos.

Dentro de los fabricantes existen gran variedad de DSP en el sector de automatización y en concreto
en la robótica, por lo que aqui se destaca los principales fabricantes y alguno de sus productos.

\textbf{Texas Instruments:} TMS320C6678
\begin{itemize}
      \item Ocho Subsistemas de Núcleos DSP TMS320C66x:
            \subitem Cada subsistema tiene un núcleo CPU C66x con frecuencias de 1.0 GHz, 1.25 GHz o 1.4 GHz.
            \subitem Rendimiento de hasta 44.8 GMAC/Core para punto fijo a 1.4 GHz y 22.4 GFLOP/Core para punto flotante a 1.4 GHz.
            \subitem Memoria por núcleo: 32K B para L1P, 32K B para L1D y 512K B para L2.

      \item Controlador de Memoria Compartida Multinúcleo (MSMC):
            \subitem 4096 KB de memoria SRAM compartida por los ocho núcleos DSP C66x.
            \subitem Unidad de Protección de Memoria para la memoria SRAM MSM y DDR3\_EMIF.

            \newpage

      \item Navigator Multinúcleo:
            \subitem 8192 colas de hardware multipropósito con administrador de colas.
            \subitem DMA basado en paquetes para transferencias sin sobrecarga.

      \item Coprocesador de Red:
            \subitem Acelerador de paquetes para soporte de IPsec, GTP-U, SCTP, PDCP y más.
            \subitem Acelerador de seguridad con soporte para IPSec, SRTP, AES, SHA-1, SHA-2 y más.
            \subitem Acelerador de cifrado con velocidad de hasta 2.8 Gbps.

      \item Periféricos:
            \subitem Cuatro carriles de SRIO 2.1 con velocidades de hasta 5 GBaud por carril.
            \subitem PCIe Gen2 con un puerto que admite 1 o 2 carriles.
            \subitem HyperLink para conexiones con otros dispositivos de la arquitectura Keystone, con soporte de hasta 50 Gbaud.
            \subitem Subsistema de interruptores Gigabit Ethernet (GbE) con dos puertos SGMII.
            \subitem Interfaz DDR3 de 64 bits con espacio de memoria direccionable de 8 GB.
            \subitem Puertos de serie de telecomunicaciones (TSIP), UART, I2C, SPI, y más.

      \item Temperaturas de Operación:
            \subitem Temperatura comercial: 0°C a 85°C.
            \subitem Temperatura extendida: -40°C a 100°C.

      \item Precio aproximado: 347,78\euro.
\end{itemize}

\textbf{Qualcomm:} QRB5165
\begin{itemize}
      \item CPU Kryo 585 de 64 bits.
            \subitem 4 núcleos Kyro Gold de alto rendimiento con caché L3 de 4MB
            \subsubitem Tres núcleos Kyro Gold con cache L2 de 256 KB por núcleo a una Fmax de 2.42 GHz.
            \subsubitem Un núcleo principal Kyro Gold con caché L2 de 512 KB, a una Fmax de 2.842 GHz.
            \subitem 4 núcleos Kyro Silver de bajo consumo con caché L2 de 128 KB por núcleo a una Fmax de 1.805 GHz.
      \item DSP Hexagon 698.
            \subitem Hexagon Vector eXtensions (quad-HVX)
            \subitem Hexagon Coprocessor (Hexagon CP) 2.0

      \item GPU Adreno 650.
            \subitem Frecuencia máxima de 587 MHz.
            \subitem Soporte para OpenGL ES 3.2, Vulkan 1.1, DX12 y OpenCL 2.0 full profile.

      \item Qualcomm 480 ISP.
            \subitem Soporte hasta 12 cámaras mediante D-PHY
            \subitem Soporte hasta 18 cámaras mediante C-PHY
            \subitem Resolución de entrada del sensor en tiempo real: 25 + 25 + 2 + 2 + 2 + 2 + 2

      \item Soporte de memoria:
            \subitem Cuatro canales de memoria de alta velocidad PoP LPDDR5 SDRAM (4 x 16-bit) diseñados para velocidades de hasta 2750 MHz de reloj y sistema de caché.
            \subitem Soporte para memoria integrada UFS 3.1 gear 4.

      \item Conectividad
            \subitem Interfaz UART, I2C, SPI, GPIO, USB 3.1, PCIe 3.0, I2S, SLIMBus, SPMI, y más.
            \subitem Soporte para Wi-Fi 6, Bluetooth 5.1 y más.

      \item Temperaturas de Operación:
            \subitem Temperatura industrial: -30°C a 105°C.

      \item Precio aproximado: 800\euro.
\end{itemize}

\textbf{Analog Devices:} ADSP-SC589
\begin{itemize}
      \item Procesador ARM Cortex-A5 Core
            \subitem Up to 500 MHz por núcleo SHARC+.
            \subitem 32 kB de caché de instrucciones L1/32 kB de caché de datos L1.
            \subitem 256 kB de caché de nivel 2 (L2) con paridad.
      \item Arquitectura Super Hardvard.
      \item Núcleos SHARC+.
            \subitem Hasta 500 MHz por núcleo SHARC+.
            \subitem Hasta 5Mb (640Kb) de memoria SRAM L1 por núcleo con paridad.
            \subitem Soporte de coma flotante de 32, 40, y 64 bits, y punto fijo de 32 bits.
            \subitem Direccionamiento por byte, palabra corta, palabra y palabra larga.
      \item Memoria.
            \subitem SRAM L2 en chip con protección ECC hasta 256KB.
            \subitem ROM L2 en chip hasta 512KB.
            \subitem DDR3/DDR3L hasta 1GB.
      \item Precio aproximado: 40\euro.
\end{itemize}

\newpage

\chapter{Conclusión}
El DSP TMS320C6678 de Texas Instruments es el elegido para la automatización industrial 
debido a su alto rendimiento, conectividad avanzada, y precio asequible de 347,78\euro. 
Este procesador incluye ocho núcleos DSP C66x y varios periféricos adicionales.

El DSP QRB5165 de Qualcomm, aunque más caro (800\euro), es ideal para aplicaciones que 
requieren alto rendimiento y conectividad avanzada, con un CPU Kryo 585 de 64 bits, 
un DSP Hexagon 698, una GPU Adreno 650, y soporte para memoria LPDDR5 SDRAM y UFS 3.1.

El DSP ADSP-SC589 de Analog Devices es una opción económica, con un precio de 40\euro, 
y ofrece un rendimiento sólido con un procesador ARM Cortex-A5 Core y núcleos SHARC+.

En resumen, el TMS320C6678 es la mejor opción general, mientras que el QRB5165 y el 
ADSP-SC589 son alternativas válidas según las necesidades específicas del proyecto.

\newpage

\chapter{Bibliografía} % En formato APA
\begin{enumerate}
      \item \url{https://odin.fi-b.unam.mx/labdsp/files/ADSP/apuntes/dsp_apli0_17.pdf}
      \item \url{https://www.ti.com/lit/ds/symlink/tms320c6678.pdf?ts=1706410966216}
      \item \url{https://www.mouser.es/c/semiconductors/embedded-processors-controllers/digital-signal-processors-controllers-dsp-dsc/?q=TMS320C6678&m=Texas%20Instruments&series=TMS320C6678}
      \item \url{https://www.thundercomm.com/product/qualcomm-robotics-rb5-development-kit/}
      \item \url{https://www.qualcomm.com/content/dam/qcomm-martech/dm-assets/documents/qrb5165-soc-product-brief_87-28730-1-b.pdf}
      \item \url{https://docs.qualcomm.com/bundle/publicresource/topics/80-PV086-1/introduction.html}
      \item \url{https://www.alldatasheet.com/datasheet-pdf/pdf/1388630/AD/ADSP-SC584.html?gad_source=1&gclid=CjwKCAjw65-zBhBkEiwAjrqRMDMgfdmVG4eEu8sXP37Q8HXxy_zamp3-L_mxcFnftncLqQ0oUCFjOhoCTzYQAvD_BwE}
\end{enumerate}

\end{document}
