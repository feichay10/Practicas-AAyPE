\documentclass[11pt]{report}

% Paquetes y configuraciones adicionales
\usepackage{graphicx}
\usepackage[export]{adjustbox}
\usepackage{caption}
\usepackage{float}
\usepackage{titlesec}
\usepackage{geometry}
\usepackage[hidelinks]{hyperref}
\usepackage{titling}
\usepackage{titlesec}
\usepackage{parskip}
\usepackage{wasysym}
\usepackage{tikzsymbols}
\usepackage{fancyvrb}
\usepackage{xurl}
\usepackage{hyperref}
\usepackage{subcaption}
\usepackage{multicol}
\usepackage{listings}
\usepackage{xcolor}
\usepackage{eurosym} % Paquete necesario
\usepackage[spanish]{babel}

\newcommand{\subtitle}[1]{
  \posttitle{
    \par\end{center}
    \begin{center}\large#1\end{center}
    \vskip0.5em}
}

% Configura los márgenes
\geometry{
  left=2cm,   % Ajusta este valor al margen izquierdo deseado
  right=2cm,  % Ajusta este valor al margen derecho deseado
  top=3cm,
  bottom=3cm,
}

% Configuración de los títulos de las secciones
\titlespacing{\section}{0pt}{\parskip}{\parskip}
\titlespacing{\subsection}{0pt}{\parskip}{\parskip}
\titlespacing{\subsubsection}{0pt}{\parskip}{\parskip}

% Redefinir el formato de los capítulos y añadir un punto después del número
\makeatletter
\renewcommand{\@makechapterhead}[1]{%
  \vspace*{0\p@} % Ajusta este valor para el espaciado deseado antes del título del capítulo
  {\parindent \z@ \raggedright \normalfont
    \ifnum \c@secnumdepth >\m@ne
        \huge\bfseries \thechapter.\ % Añade un punto después del número
    \fi
    \interlinepenalty\@M
    #1\par\nobreak
    \vspace{10pt} % Ajusta este valor para el espacio deseado después del título del capítulo
  }}
\makeatother

% Configura para que cada \chapter no comience en una pagina nueva
\makeatletter
\renewcommand\chapter{\@startsection{chapter}{0}{\z@}%
    {-3.5ex \@plus -1ex \@minus -.2ex}%
    {2.3ex \@plus.2ex}%
    {\normalfont\Large\bfseries}}
\makeatother

% Configurar los colores para el código
\definecolor{codegreen}{rgb}{0,0.6,0}
\definecolor{codegray}{rgb}{0.5,0.5,0.5}
\definecolor{codepurple}{rgb}{0.58,0,0.82}
\definecolor{backcolour}{rgb}{0.95,0.95,0.92}

% Configurar el estilo para el código
\lstdefinestyle{mystyle}{
  backgroundcolor=\color{backcolour},   
  commentstyle=\color{codegreen},
  keywordstyle=\color{magenta},
  numberstyle=\tiny\color{codegray},
  stringstyle=\color{codepurple},
  basicstyle=\ttfamily\footnotesize,
  breakatwhitespace=false,         
  breaklines=true,                 
  captionpos=b,                    
  keepspaces=true,                 
  numbers=left,                    
  numbersep=5pt,                  
  showspaces=false,                
  showstringspaces=false,
  showtabs=false,                  
  tabsize=2
}

%==============================================================================
% Cosas para la documentación LateX
% % Sangría
% \setlength{\parindent}{1em}Texto

% % Quitar sangría
% \noindent

% % Punto
% \CIRCLE \ \ \textbf{Texto} \emph{algo}
% \begin{itemize}
%   \item \textbf{Negrita:} Texto
%   \item \textbf{Negrita:} Texto
% \end{itemize}

% % Introducir código
% \begin{center}
%   \begin{BVerbatim}
%     ... Código
%   \end{BVerbatim}
% \end{center}

% Poner una imagen
% \begin{figure}[H]
%   \centering
%   \includegraphics[scale=0.55]{img/}
%   \caption{Exportación de la base de datos en formato sql}
%   \label{fig:exportación de la base de datos en formato sql}
% \end{figure}

% Poner dos imágenes
% \begin{figure}[H]
%   \begin{subfigure}{0.5\textwidth}
%     \centering
%     \includegraphics[scale=0.45]{img/}
%     \caption{Texto imagen 1}
%   \end{subfigure}%
%   \begin{subfigure}{0.5\textwidth}
%     \centering
%     \includegraphics[scale=0.45]{img/}
%     \caption{Texto imagen 2}
%   \end{subfigure}
%   \caption{Texto general}
% \end{figure}

% % Poner una tabla
% \begin{table}[H]
%   \centering
%   \begin{tabular}{|c|c|c|c|}
%     \hline
%     \textbf{Campo 1} & \textbf{Campo 2} & \textbf{Campo 3} & \textbf{Campo 4} \\ \hline
%     Texto & Texto & Texto & Texto \\ \hline
%     Texto & Texto & Texto & Texto \\ \hline
%     Texto & Texto & Texto & Texto \\ \hline
%     Texto & Texto & Texto & Texto \\ \hline
%   \end{tabular}
%   \caption{Nombre de la tabla}
%   \label{tab:nombre de la tabla}
% \end{table}

% % Poner codigo de un lenguaje a partir de un archivo
% \lstset{style=mystyle}
% The next code will be directly imported from a file
% \lstinputlisting[language=Python]{code.py}

% “Texto entre comillas dobles”

% • 

%==============================================================================

\begin{document}

% Portada del informe
\title{Actividad sobre DSP}
\subtitle{Arquitecturas Avanzadas y de Propósito Específico}
\author{Cheuk Kelly Ng Pante (alu0101364544@ull.edu.es)}
\date{\today}

\maketitle

\pagestyle{empty} % Desactiva la numeración de página para el índice

% Índice
\tableofcontents

% Nueva página
\cleardoublepage

\pagestyle{plain} % Vuelve a activar la numeración de página
\setcounter{page}{1} % Reinicia el contador de página a 1

% Secciones del informe
% Capitulo 1
\chapter{Introducción a los DSP}
Un procesador de señales digitales o DSP es un sistema basado en un procesador o
microprocesador que posee un conjunto de instrucciones, un hardware y un software
optimizado para aplicaciones que requieren operaciones numéricas a muy alta velocidad.

Algunos fabricantes de DSP son:
\begin{multicols}{2}
  \begin{itemize}
    \item Analog Devices
    \item Texas Instruments
    \item NXP
    \item STMicroelectronics
  \end{itemize}

  \columnbreak

  \begin{itemize}
    \item Infineon
    \item Qualcomm
    \item Broadcom
  \end{itemize}
\end{multicols}

% Capitulo 2
\chapter{Automatización industrial}
La automatización industrial es la aplicación de tecnologías para operar procesos
y maquinaria sin la intervención humana. Se utiliza en una gran variedad de industrias,
como la automotriz, la robótica, la aeroespacial, la naval, la de empaquetado, la de
alimentos y bebidas, la de productos farmacéuticos y la de fabricación de metales.

Algunas aplicaciones de la automatización industrial son: control de procesos, robots industriales y colaborativos,
sistemas de transporte, monitoreo y control, sistemas de prueba, sistemas de seguridad, etc.

\section{DSP en la automatización industrial}
Los DSP (Procesadores Digitales de Señal) son dispositivos que realizan operaciones matemáticas
sobre señales digitales, como audio, vídeo o datos. Se utilizan en muchos campos de la automatización
industrial, como el control de procesos, la comunicación, el filtrado o la conversión de datos.

Dentro de los fabricantes existen gran variedad de DSP en el sector de automatización y en concreto
en la robótica, por lo que aqui se destaca los principales fabricantes y sus productos.

\textbf{Texas Instruments:} TMS320C6678
\begin{itemize}
  \item Ocho Subsistemas de Núcleos DSP TMS320C66x:
        \subitem Cada subsistema tiene un núcleo CPU C66x con frecuencias de 1.0 GHz, 1.25 GHz o 1.4 GHz.
        \subitem Rendimiento de hasta 44.8 GMAC/Core para punto fijo a 1.4 GHz y 22.4 GFLOP/Core para punto flotante a 1.4 GHz.
        \subitem Memoria por núcleo: 32K B para L1P, 32K B para L1D y 512K B para L2.

  \item Controlador de Memoria Compartida Multinúcleo (MSMC):
        \subitem 4096 KB de memoria SRAM compartida por los ocho núcleos DSP C66x.
        \subitem Unidad de Protección de Memoria para la memoria SRAM MSM y DDR3\_EMIF.

        \newpage

  \item Navigator Multinúcleo:
        \subitem 8192 colas de hardware multipropósito con administrador de colas.
        \subitem DMA basado en paquetes para transferencias sin sobrecarga.

  \item Coprocesador de Red:
        \subitem Acelerador de paquetes para soporte de IPsec, GTP-U, SCTP, PDCP y más.
        \subitem Acelerador de seguridad con soporte para IPSec, SRTP, AES, SHA-1, SHA-2 y más.
        \subitem Acelerador de cifrado con velocidad de hasta 2.8 Gbps.

  \item Periféricos:
        \subitem Cuatro carriles de SRIO 2.1 con velocidades de hasta 5 GBaud por carril.
        \subitem PCIe Gen2 con un puerto que admite 1 o 2 carriles.
        \subitem HyperLink para conexiones con otros dispositivos de la arquitectura Keystone, con soporte de hasta 50 Gbaud.
        \subitem Subsistema de interruptores Gigabit Ethernet (GbE) con dos puertos SGMII.
        \subitem Interfaz DDR3 de 64 bits con espacio de memoria direccionable de 8 GB.
        \subitem Puertos de serie de telecomunicaciones (TSIP), UART, I2C, SPI, y más.

  \item Temperaturas de Operación:
        \subitem Temperatura comercial: 0°C a 85°C.
        \subitem Temperatura extendida: -40°C a 100°C.

  \item Precio: 347,78\euro. 
\end{itemize}

\textbf{Qualcomm:} RB5

\textbf{Analog Devices:} ADSP-SC589 

\newpage

\chapter{Bibliografía} % En formato APA
\begin{enumerate}
  \item Ng Pante, C. (2001). Titulo. Nombre pagina web. Recuperado de \url{http://url.com}
  \item \url{https://odin.fi-b.unam.mx/labdsp/files/ADSP/apuntes/dsp_apli0_17.pdf}
  \item \url{https://www.ti.com/lit/ds/symlink/tms320c6678.pdf?ts=1706410966216}
  \item \url{https://www.mouser.es/c/semiconductors/embedded-processors-controllers/digital-signal-processors-controllers-dsp-dsc/?q=TMS320C6678&m=Texas%20Instruments&series=TMS320C6678}

\end{enumerate}

\end{document}
